1-What does the line counter[item] = 1 do in the code below?

for item in items:
  if (item in counter.keys():
  counter[item] +=1
  else:
  counter[item]=1
 
answer: It creates a dictionary entry for item and sets its value to one.

2-Finding the maximum value recursively has the same time complexity as _____.


answer: an unordered list search

3-Which result happens when you add a redundant entry to a hash table?


answer: The new entry replaces the old one.

4-Which line in the code snippet below adds content to a hash table?
1 items = ["apple", "pear", "orange", "banana", "apple","orange", "apple", "pear", "banana", "orange","apple", "kiwi", "pear", "apple", "orange"]

2 filter = dict()

3 for key in items:
4       filter[key] = 0

5 result = set(filter.keys())
6       print(result)


answer: 3

5-Evaluate the code snippet below.
At what point does recursion occur when using an algorithm to find the highest value in a data set? 

1 def find_max(items):
2     if len(items) == 1:
3         return items[0]
4 op1 = items[0]
5 print(op1)
6 op2 = find_max(items[1:])
7 print(op1, op2)


answer: line 6
