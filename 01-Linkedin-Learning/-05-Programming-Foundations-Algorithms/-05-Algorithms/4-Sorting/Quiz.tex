1-Which of the following sorted datasets would be easiest for a user to understand?

answer: a list of items that are ordered by price from lowest to highest

2-How does a bubble sort find the largest value in a data set?

answer: All value pairs are compared to each other until the largest is at the top.

3-Which condition is true after the end of one recursion step in a quicksort?

answer: The pivot element is placed at its correct array location.

4-What method does a merge sort use to break down data and sort it?

answer: It uses recursion to break down data into smaller sets in order to find the appropriate order.

5-What are present in bubble-sort codes that indicate their time complexity?

answer: nested loops

6-A programmer needs to organize a data set. How will a quick sort accomplish this?

answer: It will use a pivot point and move items that are on the wrong side of the pivot value by using high and low index values.

7-How does the required time for a merge sort grow with the size of the data set?

answer: logarithmic-linear
