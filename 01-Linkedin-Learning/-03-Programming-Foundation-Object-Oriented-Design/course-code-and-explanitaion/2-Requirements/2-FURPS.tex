
FURPS 
One commonly used model for classifying software quality attributes 
FURPS serves as a checklist of several key qualities to consider when determining requirements. 

FURPS refer to 

functionality:  Functionality refers to the capabilities and features of the app. 
  capability, reusability, security 

useability:  affects the person who will be using the program.
Is it easy on the eyes?
 Is it intuitive to use? 
 Is the documentation accurate and complete? 
  Human factors, Aesthetics, Consistency, Documentation 

Reliability:
  Is it easy on the eyes? Is it intuitive to use? 
  Is the documentation accurate and complete? 

  availability, failure rate & duration, predictability

performance: 
  dictate the application's response time through put. 
  And they put limits on the system resources it can use. In supportability. 

  speed, efficiency, resource consumption, scalability 

supportability:

  Make sure the application can be tested, extended, serviced and installed and configured.

  testability, extensibility, serviceability, configurability   


FURPS+   add more four categories  

Design:
  addresses constraints on how the software must be built because the app requires certain things such as a relational database.

Implementation:
  Does it have to be written in a certain language?
  Are there standards or methodologies that need to be followed?

Interface: 
  refer to an external system that needs to be interfaced with. 

Physical requirements:
  Actual physical constraints related to the hardware the application will be deployed on.