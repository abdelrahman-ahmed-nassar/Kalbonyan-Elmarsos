1-What is the difference between object-oriented programming and procedural programming?


answer:Procedural programming specifies a sequence of tasks, but object-oriented programming describes the properties of tools or items.
 
2-How does dynamic polymorphism differ from static polymorphism?


answer: It uses overriding instead of overloading.

3-What is overriding a method?


answer: Creating a unique version of an inherited method.

4-How are analysis and design different?


answer: Analysis describes a problem; design describes a solution.

5-What is the term for a visual representation of the classes in an application?


answer: class diagram

6-In addition to attributes and methods, what does a UML class diagram contain?


answer: the class name

7-How do object behaviors and attributes differ?


answer: Attributes describe a state, but behaviors describe actions.

8-You are designing a traffic simulation program. What is a possible attribute that you could use for a car object?


answer:gas mileage

9-Shonzu has gathered the requirements for a new solution, described the application he is going to build, and identified the main objects in the solution. What should he do next?


answer: Describe object interactions.

10-What is the purpose of encapsulation?


answer: to protect an object from unwanted changes

11-In addition to attributes and behaviors, which quality must a class possess?


answer: a name

12-In the following class diagram, what does lower() represent?

TRIPOD
height
width
angle
raise()
lower()
point()
fold()


answer: a behavior

13-What is a benefit of using a programming language that has a large library?


answer: Many classes are already defined and can be used without have to re-define them.

14-In the following class diagram, what does height represent?

TRIPOD
height
width
angle
raise()
lower()
point()
fold()

answer: an attribute

15-We're using abstraction when we define a(n) _____.


answer: class

16-Focusing on the idea of a person instead of an individual is an example of what fundamental idea in object-oriented programming?


answer:abstraction
 
17-Steve is able to turn on and adjust his television even though he does not know how it works internally. This exemplifies which principle of object-oriented programming?


answer: encapsulation

18-Why is inheritance used when creating a new class?


answer: to avoid redefining attributes and behaviors

19-If an attribute is added to a superclass, what happens to all of the objects of the subclass?


answer: Each subclass object automatically receives the additional attribute.

20-Static polymorphism uses method _____.

answer:overloading