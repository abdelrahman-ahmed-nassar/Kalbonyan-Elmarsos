1-What is the difference between git diff and git diff --staged?

answer: git diff compares the changes to working directory files to the staging index, while git diff --staged compares staged files to the repository versions.

2-Which of these methods is best to deal with a file called this_file.txt that should NOT be deleted but should no longer be referenced when git status is used?

answer: Move this_file.txt from the project directory and use git rm this_file.txt to remove the file.

3-How is staging an edited file different from staging a new file?

answer: Both require git add to stage them.

4-If the output of git status reveals that file file1.php is untracked, what is true about the file?

answer: There will be no record of further changes made on file1.php.

5-If a user must use git reset HEAD tabs.js to unstage tabs.js, which command likely created the need to reset the HEAD?

answer: git add .

6-The output of git status is the following text, the last line in green: Changes to be committed:

 (use "git reset HEAD <file>..." to unstage)

 deleted: file1.php

 
Which action did NOT create this result?

answer: file1.php was deleted with an operating system tool rather than git rm file1.php

7-How does committing an edited file affect the Git architecture's trees?

answer: The commit moves the file from the staging index to the repository.

8-What is the most efficient way to rename a file in a repository?

answer: Use git mv with the file name.

9-In which situation should you use git diff?

answer: to observe specific changes from the original version of a file

10-What does the snippet of git diff output below signify? Assume these output lines are contiguous.

- background: #78b210 url("bg.png");
+ background-color: #d0f0f6;

answer: The edit changed the line by replacing the first line with the second.
