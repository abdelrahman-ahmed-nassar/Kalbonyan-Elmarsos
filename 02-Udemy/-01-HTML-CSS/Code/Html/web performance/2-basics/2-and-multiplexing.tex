- [Instructor] Browsers and servers use the HTTP protocol to talk to each other, and
send and receive files and data. In fact, every interaction happening between the 
browser and the server is an HTTP transaction over this protocol. When you interact
with content on the web today, you're using one of two different versions of the 
HTTP protocol, either the old HTTP/1.1 or the more modern HTTP/2. Which protocol 
version is in use has a significant impact on the performance of the site. Let me 
explain. In HTTP/1.1, all files requested by the browser are loaded synchronously, 
one after the other. So a typical HTML page with two style sheets, a couple of 
images, and some JavaScript would require the browser to first load the HTML 
, then the CSS files, then the JavaScript files, and finally the image files one 
after the other. This is slow, inefficient, and a recipe for terrible performance.
To work around this obvious issue, browsers cheat by opening up to six parallel 
connections to the server to pull down data. However, this creates what's known 
as head of line blocking, where the first file, the HTML file, holds back the 
rest of the files from downloading. It also puts enormous strain on the internet 
connection and the infrastructure, both the browser and the server, because 
you're now operating with six connections instead of one single connection. In
HTTP/2, we have what's known as multiplexing. The browser can download 
separate files at the same time over one connection, and each download is 
independent of the others. That means with HTTP/2, the browser can start 
downloading a new asset as soon as it's encountered, and the whole process 
happens significantly faster. Now, for HTTP to work, a few key conditions need 
be met. Number one, the server must support HTTP/2. Now most servers do, but 
it's worth checking. Number two, the browser must also support HTTP/2. All 
modern browsers do, but again, it's worth checking if you're working with 
really old infrastructure. And number three, the connection must be encrypted
over HTTPS. If any of these conditions are not met, the connection 
automatically falls back to HTTP/1.1. So bottom line, for instant performance
improvements with minimal work, get an SSL certificate for your domain. 
They're available for free from services like Let's Encrypt and ensure your 
server supports HTTP/2. As a bonus, your site traffic is encrypted out of 
gate, which means safer browsing for your visitors and higher ranking on 
Google and everywhere else.
