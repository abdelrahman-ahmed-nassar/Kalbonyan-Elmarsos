- [Instructor] A tried and true method for performance optimization is caching or 
storing of assets. This can be done at three main levels in the transaction between 
the browser and the server. On the server, on the CDN or in the browser itself. Your
caching strategy will depend on how your site or service works, and how the 
infrastructure of your host or CDN works. Here are the basics for all three of 
levels. If you're running a site or service relying on server-side rendering, 
meaning each page or view is generated on the fly by the server when it is 
requested, caching may provide a huge performance boost. This scenario includes 
pretty much all established content management systems like WordPress, Drupal, 
Joomla! Magento and so on. By enabling caching, the server no longer has to render
the page every time the page is requested. Instead when the page is rendered, a 
snapshot of that page is created and then stored in the server cache. The next 
time a visitor then comes to the site, there'll be handed at this stored cached 
snapshot instead of a freshly rendered page. That way the server doesn't have to
rerender the page for every single visitor. This by the way is why so-called static
site generators have become so popular in the web dev community. They produce 
pre-rendered cacheable static pages out of the box, and bypass the entire CMS 

side rendering problem. The challenge with this type of caching for these types of
services is they often have dynamic features like common sections. So every time 
a new comment is added, the cache needs to be cleared, and then the page has to 
be regenerated. Even so, caching should be enabled for all sites relying on 
server-side rendering because performance benefits are so significant. The next 
step is CDN caching. CDNs are effectively external caching services for sites. In 
their most basic configuration, they will serve cast versions of a site or 
service to the viewer, but CDNs can also do a lot more including what's known as
edge computing. Here, the rendering of a site is done by the CDN at the point of
contact with the visitor. So instead of the server rendering the page and 
delivering cache snapshot to the CDN, the CDN renders the page when requested 
then caches it itself. This edge approach works well with modern static site 
generators like Gatsby and all JavaScript based site generators and frameworks 
because they serve up static assets by default, and are built to work in this 
modern web architecture. Once that connection is made between the server and 
the CDN of the browser, and all files are being sent to the browser, we get to 
the last caching step, storing assets in the browser for future use. There are 
two main things we can do here. One, store existing assets. So if the visitor 
returns to the site it already has all the information cached in the browser 
and two, push files to the browser early so by the time the browser requests 
the file, the files that are already sitting in the cache. All browsers do 
some level of caching automatically. That's why you find yourself needing to 
refresh your browser multiple times or even go to incognito mode or clear the
cache when you're developing a site to see the most recent changes. The 
browser will always cache CSS and JavaScript and image files unless 
explicitly told not to. And we can then instruct the browser on exactly how 
we want to handle caching of our assets. For assets that are unlikely to 
change such as main style sheets, JavaScript, and other images, long caches 
makes sense. For assets that are likely to change over time, short cache 
durations, or no cashing at all may make more sense. To ensure new and 
updated assets always make it to the visitor. We can use cache busting 
strategies like appending automatic hashes to file names or something else. 
Or we can rely on the server itself to document the file name on file date 
for each file, and then do the caching automatically. There is one more 
benefits to caching that is rarely talked about but it's really important. 
Thanks to multiplexing. You can now split up CSS and JavaScript files into 
smaller modules. And that means when you update something in CSS or 
JavaScript, instead of having to recache an entire style sheet for an entire
site, you're just recaching the module that has that update. That goes for 
CSS and JavaScript and takes full advantage of all this caching that's 
happening at every step in the process.


